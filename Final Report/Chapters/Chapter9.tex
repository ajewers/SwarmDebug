% Chapter 9

\chapter[Implementation]{Implementation} % Main chapter title

\label{Chapter9} % For referencing the chapter elsewhere, use \ref{Chapter9} 

%----------------------------------------------------------------------------------------

\section{Overview}
This section gives details of the implementation phase of the project, including descriptions of how parts of the system function, information regarding the development process and explanations for some of the key decisions made regarding the implementation. 

%----------------------------------------------------------------------------------------

\section{Code Structure}
The system is implemented in two parts; a collection of C++ source and header files which make up the main application, and a smaller collection of C++ source and header files which handle the robot-side portion of the system. In addition to the source files the main application also contains a number of other files which are used by the Qt system to define the user interface and manage the build process. Table \ref{tab:CodeFiles} details the names and purposes of all files within the main application. Table \ref{tab:RobotCodeFiles} details the files that make up the robot side API.

\begin{longtable}{ l p{10cm} }
\caption[Application Code Files]{Source code and tertiary files that make up the main application.}\\
 File & Purpose\\ 
 \hline
 main.cpp & The entry point for the application. Instantiates the MainWindow class.\\
 mainwindow.cpp, .h & The core class, contains the entry point for the application and controls the set up and tear down processes and handles UI events within the main window.\\
 mainwindow.ui & Describes the user interface layout in a XML-like format. Used by the Qt framework to construct the UI.\\
 datamodel.cpp, .h & The top level class encapsulating the full data model. Maintains a list of RobotData objects.\\
 robotdata.cpp, .h & A class encapsulating the data of a single robot, including ID, position, state, sensor data and user data.\\
 datathread.cpp, .h & This class contains all routines for receiving data from the robots via wifi, and is designed to be run on a thread of its own.\\
 cameracontroller.cpp, .h & The high level class encapsulating the routines and data related to the tracking camera. This class is designed to be independent of the camera hardware being used.\\
 machinevision.cpp, .h & A lower level class encapsulating routines for interfacing with the camera hardware.\\
 visualiser.cpp, .h & This class encapsulates the visualiser GUI object, and is implemented to conform the Qt framework as a custom extension to the QWidget class. Also contains routines for applying the video augmentations as per the current visualiser settings.\\
 viselement.h & Contains an abstract class definition for a single visualiser settings element. These elements are used to define how specific elements of the video augmentation are rendered, and also contain settings and variables relevant to this task. \\
 vis*.cpp, .h, .c & Classes beginning with the `\textit{vis}' prefix derive from the VisElement abstract class and contain routines for rendering the visualisation for one type of data. The latter part of the class name identifies which data type is targeted.\\
 irdataview.cpp, .h & A custom GUI object, derived from QWidget, which displays the raw IR sensor data as a bar graph in the data window.\\
 settings.cpp, .h & Encapsulates the general application settings and routines for changing their values according to user input.\\
 log.cpp, .h & Encapsulates the routines for logging events and data to text files.\\
 util.cpp, .h & Contains static utility functions used in various places throughout the application code.\\
 *settingsdialog.cpp, .h & Classes ending with the `\textit{settingsdialog}' suffix describe dialog windows for adjusting the settings related to the visualisation of specific data types, identified by the first part of the class name.\\
 appconfig.h & Contains pre-processor definitions for controlling inclusion/exclusion of code segments.\\
 SwarmDebug.pro & Used by the Qt framework to build the application. Directs to the necessary code files and libraries.\\
 \bottomrule\\
	
 \label{tab:CodeFiles}
\end{longtable}

\begin{longtable}{ l p{10cm} }
\caption[Robot-side Code Files]{Source code files that make up the robot side API.}\\
 File & Purpose\\
 \hline
 debug\_network.cpp & This file encapsulates networking functionality for communicating with the debugging system. Contains routines for sending data of specific types, as well as for sending raw packets.\\
 debug\_network.h & Header file for the debugging system network interface. Also contains definitions for data type identifiers.\\
 \bottomrule\\
	
 \label{tab:RobotCodeFiles}
\end{longtable}

%----------------------------------------------------------------------------------------

\section{Tracking System Integration} \label{TrackingSystemIntegration}


%----------------------------------------------------------------------------------------

\section{Networking}


%----------------------------------------------------------------------------------------

\section{Data Transfer Formats}
In order to retrieve data from the robots in a usable fashion a common format for exchanging data needed to be defined, and used by both ends of the communication link. A number of different options were considered for achieving this, ranging from super-lightweight custom packets using the minimum number of bytes to established existing solutions such as the JSON data interchange standard.

%----------------------------------------------------------------------------------------

\section{Data Model}


%----------------------------------------------------------------------------------------

\section{Visualiser}


%----------------------------------------------------------------------------------------

\section{Visualiser}


%----------------------------------------------------------------------------------------

\section{Robot Code}


%----------------------------------------------------------------------------------------