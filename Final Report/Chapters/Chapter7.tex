% Chapter 7

\chapter[Initial User Survey]{Initial User Survey} % Main chapter title

\label{Chapter7} % For referencing the chapter elsewhere, use \ref{Chapter6} 

%----------------------------------------------------------------------------------------

\section{Overview}
In order to determine how best to implement the system in order to be useful in practice a survey waas carried out at the YRL. Those asked were all actively engaged in robotics work, either in a research capacity or as a technician. The survey aimed to determine first if a level on interest existed for such a system, and then which specific features were most desired. This would go on to influence design choices and inform priorities during development.

%----------------------------------------------------------------------------------------

\section{Data}
The questions and results of the survey are presented here.

\textbf{Question 1: Do you believe that a system for displaying internal robot data for a swarm of robots in real time would be useful when debugging swarm robotics behaviours and/or conducting swarm robotics experiments?}

\begin{center}
\begin{tabular}{ c c c }
 Answer & Votes & Percentage \\ 
 \hline
 Yes & 4 & 80 \\ 
 No & 1 & 20 \\
 No Opinion & 0 & 0 \\
\end{tabular}
\end{center}

\vspace{1cm}

\textbf{Question 2: Do you believe that such a system would benefit from the inclusion of an 'augmented reality' component - whereby data retrieved from the robots could be displayed in graphical forms, overlaid on a live video feed of the robots themselves?}

\begin{center}
\begin{tabular}{ c c c }
 Answer & Votes & Percentage \\ 
 \hline
 Yes & 5 & 100 \\ 
 No & 0 & 0 \\
 No Opinion & 0 & 0 \\
\end{tabular}
\end{center}

\vspace{2cm}

\textbf{Question 3: Please rate each of the following potential features based on your opinion of their importance or usefulness to the system proposed, on a scale of one to five, where one indicates a feature is not important or useful, and five indicates a feature is very important or useful.}

On the questionnaire the values of one to five were qualified as `\textit{Not important or useful}', `\textit{Unlikely to be important or useful}', `\textit{Neutral}', `\textit{Somewhat important or useful}' and `\textit{Very important or useful}' respectively.

\begin{center}
\begin{tabular}{ p{10cm} c c c c c }
 Feature & 1 & 2 & 3 & 4 & 5 \\ 
 \hline
 Real time video feed of the robots and their environment & 0 & 0 & 0 & 0 & 5\\
 Augmentation of the video feed with the position and orientation of each robot & 0 & 0 & 1 & 1 & 3\\
 Augmentation of the video feed with each robots identifier (ID or name) & 0 & 0 & 0 & 1 & 4\\
 Augmentation of the video feed with spatially situated sensor data, represented in a graphical form & 0 & 0 & 0 & 2 & 3\\
 The ability to enable and disable individual elements of the video feed augmentation & 0 & 0 & 0 & 2 & 3\\
 The ability to customise individual elements of the video feed augmentation (colour, size, etc.) & 0 & 0 & 2 & 1 & 2\\
 Displaying the robots internal state and a history of recent state transitions & 0 & 0 & 0 & 3 & 2\\
 Displaying raw sensor data (e.g. IR) in textual format & 0 & 0 & 0 & 4 & 1\\
 Displaying sensor data in plotted graph formats & 0 & 0 & 1 & 3 & 1\\
 Support for displaying unspecified user data in textual form & 0 & 0 & 1 & 2 & 2\\
 The ability to compare two or more specific robots within the swarm & 0 & 1 & 1 & 1 & 2\\
 Logging received data and events to a text or CSV file & 0 & 0 & 0 & 1 & 4\\
\end{tabular}
\end{center}

\vspace{1cm}

\textbf{Question 4: Please briefly describe any additional features you believe would be useful, based on your experiences working with swarm robotics systems}

\begin{itemize}
\item ``macro-level behavioural data on the swarm; e.g., number of robots in different behavioural states (color coded accordingly).''
\item ``I think the system (already potentially very useful) could be improved/expanded further to add the option of more post-processing/extraction of data. The ability to create video of the over-lay, and perform statistical analysis on the complete run, would begin to turn the system in not only a very useful debugging system but a complete package allow researchers to gather high-quality, publication ready analysis of swarm experiments.''
\end{itemize}

\vspace{2cm}

\textbf{Question 5: Which of the following aspects of the robot data do you believe should be made available by the application to aid in debugging and testing? Tick all that apply. Please add any more you may think of.}

\begin{center}
\begin{tabular}{ p{10cm} c }
 Feature & Votes\\ 
 \hline
 Position & 4\\
 Orientation / direction & 4\\
 Position change over time (recent path tracking) & 4\\
 Internal state machine state & 3\\
 Internal state transition history & 3\\
 IR sensor values & 4\\
 Distance between robots & 3\\
 Robot ID & 4\\
 Other & 2\\
\end{tabular}
\end{center}

Additional responses:

\begin{itemize}
\item ``Option to include user-defined data so if a certain controller implements a timer that facilitates a state transition might be useful to see the value of that timer whilst debugging to compare with state transitions''
\item ``A simple API to add user-defined variables/statuses''
\end{itemize}

\vspace{1cm}

\textbf{Please add any additional comments you have about the proposed application in relation to your experiences working with swarm robotics systems}

\begin{itemize}
\item ``A client-server model between back-end (camera) and remote client would potentially be very useful; also the system should be flexible and not reliant on a specific camera/server setup. Would be very interesting to see if it will work on a R-Pi 3 + camera combination, as this would allow for portable tracking setups.''
\item ``All real-time information is useful!''
\end{itemize}

%----------------------------------------------------------------------------------------

\section{Analysis}
The response to question one indicates a reasonable level of interest in the system amongst those surveyed. 

%----------------------------------------------------------------------------------------