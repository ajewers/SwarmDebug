% Chapter Project Plan

\chapter[Project Plan]{Project Plan} % Main chapter title

\label{ChapterPlan} % For referencing the chapter elsewhere, use \ref{Chapter1} 

%----------------------------------------------------------------------------------------

This project was completed between Monday 16th January and Thursday 18th May, 2017. A well-defined breakdown of the tasks required to complete the project, and an organised plan for completing these tasks, was instrumental in ensuring that the project could be completed in the available time. This chapter gives details of this work breakdown, and the timing considerations. It also includes details of a number of potential risks identified at the start of the project, and information on how these were mitigated where possible. It should be noted that a significant majority of work involved in this project was software development based, and - as with almost all modern software development - accurately predicting the time required to implement every piece of code was a virtually impossible task. The development process inevitably leads to the discovery of unforeseen issues and a deeper understanding of the problem constraints, which in turn requires the allocation of project time to be re-evaluated. Hence wherever possible an `\textit{agile}' methodology and approach was employed. This included frequent re-assessment of the remaining work and the feasibility of individual features. The agile methodology, and its application within this project, is discussed in more detail in section \ref{Agile}.

%----------------------------------------------------------------------------------------

\section{Work Breakdown}
At the start of the project the software related work, including both development and testing stages, was divided into logical tasks. These tasks are shown in tables \ref{tab:DevTasks} and \ref{tab:TestingTasks} respectively. The time required to complete each task was estimated based on prior experience of software development work, and these approximate timings are listed in the tables. Other tasks, not related specifically to the software development, are listed in table \ref{tab:OtherTasks}.

\clearpage
\begin{longtable}{ >{\raggedright}p{5cm}>{\raggedright}p{6cm}p{3cm} }
	\caption{Development tasks.}\\
	\toprule
	\tabhead{Task} & \tabhead{Objective} & \tabhead{Approximate Time} \\
	\midrule
	
Read and Understand Existing Code & To understand existing code related to the tracking camera and networking on the e-pucks. & 14 Days (Alongside other development)
\\
Establish Development Environment and Toolchain & To enable organised development by establishing a tool set and workflow. & 3 Days 
\\
Learn to Re-Program e-puck Robots & To understand the cross compilation process for the e-pucks. & 2 Days 
\\ 
Outline Software Architecture & To design a coherent code structure in order for code to remain organised and modular. & 2 Days 
\\
Design General User Interface & To create a high level design of the basic UI and implement a skeleton framework of this UI. & 3 Days
\\ 
Incorporate Tracking Camera Code & To incorporate existing low-level code for acquiring images from the tracking camera and performing tag detection. & 2 Days
\\
Implement Tracking Camera Controller & To implement code to create a layer of abstraction between the application code and the tracking code. & 2 Days 
\\
Implement Wireless Data Receive & To implement code to allow the application to receive data wirelessly. & 3 Days
\\
Determine Robot Data Types & To establish an initial set of data types that will be supported by default, and a packet format for these. & 2 Days 
\\
Design and Implement Data Model & To design the back end data model of the application and implement it in code. & 6 Days 
\\
Implement Mapping Received Data to Model & To implement code to store received robot and tracking data in the application data model. & 3 days 
\\
Implement Basic Visualiser & To implement code for displaying the video feed and augmenting it with basic geometric primitives. & 5 Days 
\\
Design UI Data Representation & To establish a design for the representation of the different data types. & 2 Days
\\
Implement Graphical and Textual Data Visualisation & To implement code to convert the data in the data model into relevant visualisations. & 10 Days 
\\
Implement Data Visualisation Filtering & To implement code to allow the user to filter out unnecessary visualisation elements. & 5 Days 
\\
Implement Robot Data Comparison & To implement code to allow the user to compare the data of specific robots. & 3 Days
\\
	\bottomrule\\
	\label{tab:DevTasks}
\end{longtable}

\begin{longtable}{ >{\raggedright}p{5cm}>{\raggedright}p{6cm}p{3cm} }
	\caption{Testing tasks.}\\
	\toprule
	\tabhead{Task} & \tabhead{Objective} & \tabhead{Approximate Time} \\
	\midrule
	
Continuous Integration Testing & To continually test newly implemented features with the system as a whole during the implementation process. & Throughout development
\\
Verification Testing & To verify the correct operation of the software through formal testing, with a specific focus on the data model and the user interface. & 10 Days
\\
Verification Fixes and Changes & To make the necessary changes to correct issues identified in the verification testing. & 5 Days
\\ 
Final Fixes and Changes & Some leeway time is available to make any final changes or fixes based on the results of the user evaluation sessions. & Remaining time 
\\ 
	\bottomrule\\
	\label{tab:TestingTasks}
\end{longtable}

\clearpage
\begin{longtable}{ p{5cm}p{9cm} }
	\caption{Other tasks.}\\
	\toprule
	\tabhead{Task} & \tabhead{Objective} \\
	\midrule
	
Initial Report & To produce an initial report in the early stages of the project, outlining the preliminary research completed and the project plan at this stage.
\\
Create Initial User Survey & To create a survey to be answered by potential users of the system such as robotics researchers to gauge interest levels for the proposed system and specific features.
\\
Distribute Initial User Survey and Collate Results & To distribute the survey to swarm robotics researchers and others with relevant experience, and collate and analyse the responses.
\\
Create a System Evaluation and User Testing Plan & To devise a plan for evaluating the effectiveness of the system, including a detailed description of the user testing procedure.
\\
User Evaluation Sessions & To evaluate the system by allowing a number of users to use it in a structured evaluation session.
\\
	\bottomrule
	\label{tab:OtherTasks}
\end{longtable}

%----------------------------------------------------------------------------------------

\section{Timing and Plan}

Having estimated the time required for each development task, a plan for completing the project within the available time frame was devised, and can be found as a Gantt chart in appendix \ref{AppendixGantt}. The plan comprised three main phases; a preliminary phase of research and design work, the main implementation phase, and a final testing and evaluation phase. Where possible, tasks were organised to leave weekend days free, in an attempt to provide a more manageable schedule, but also to allow some slippage time each week for any overrunning tasks.

%----------------------------------------------------------------------------------------

\section{Risk Analysis and Mitigation}

Engineering projects in all fields are subject to a wide range of risks which may prevent their successful completion. A common technique to reduce the chances of an unsuccessful project is to analyse the potential risks involved in the project before work begins, and attempt to mitigate these risks wherever possible. Risks with a high likelihood of occurring or a high impact on the project are prioritised for mitigation first. For this project the following risks were identified, and mitigation steps taken.

\noindent\textbf{1. Failure to complete the work in the available time.}

The primary source of risk within this project was time. Software development is an inherently time consuming process, with most implementation tasks suffering from a degree of uncertainty. The time frame for this project was also relatively short. Therefore a major risk to be considered was the potential for features not to be implemented due to a lack of available time, and for the system to fall short of its stated aims and objectives as a result. A number of steps were taken to mitigate this risk. Firstly, the features of the system required to satisfy the core aims were assessed, and the task schedule was organised such that a `\textit{minimum viable product}' (MVP) would be completed as quickly as possible. The MVP describes the smallest possible set of features which can be implemented such that the system still satisfies its core objectives. The second step taken to mitigate the time-risk was the inclusion of slippage time in the project plan. This extra time was worked into the plan to absorb any overrun in individual tasks, reducing the chances of a single overrunning task having a knock-on effect on the rest of the plan. Slippage time was introduced in two places; by arranging tasks to leave weekend days free, as previously discussed, but also by planning for a flexible testing phase, where time previously allocated for testing could be re-allocated for development if this was deemed necessary. The knock-on affect of reduced testing was determined to be an undesirable but acceptable consequence.

\noindent\textbf{2. Loss of development computer due to a hardware issue.}

Another source of potential risk in this project was the development hardware. The majority of the development work was carried out on a personal laptop, running a Linux operating system in a virtual environment. Any damage to this computer could have led to the loss of important code, and interrupted the development process while a new machine was found. In order to mitigate the potential risk of code loss all code was stored in an online repository, and the latest code was uploaded to this repository frequently. The development environment required to compile the project code was also set up on the tracking server which would be running the final application. This was done so that the code could be tested on the target environment during development, but this also partially mitigated the timing risk involved in the loss of the main development machine, as work could be continued temporarily on the tracking server while a replacement machine was found and set up.

\noindent\textbf{3. Loss of key system component due to a hardware issue.}

All electronic hardware is susceptible to breakage, faults and malfunctions. The system developed in this project includes a number of hardware elements working in tandem, and the loss of any single component could have halted the implementation and testing phases of the project until a replacement could be found. In the case of the robots this risk was mitigated by the fact that a reasonable number of robots were available, and a broken or malfunctioning robot could be replaced without much disruption. This was especially true if the Linux extension board was still operating correctly, as this could easily be moved to another standard e-puck. Mitigation by redundancy in this manner was however not an option for some of the larger, more expensive components of the system. At the time of the project the YRL possessed only one tracking camera. Had it suffered a fault or been damaged this would have disrupted the project greatly whilst it was fixed or replaced. However, due to the fact that the tracking camera required no moving parts, was not moved during the course of the project, and had shown no signs of unreliability in the past, the chances of a fault occurring or the camera being damaged were considered to be quite low. Therefore, although this was a potentially high-impact risk, its low-probability meant that it could be accepted without mitigation. The tracking server itself was considered more likely to suffer a fault, and this risk was partially mitigated by the fact that another server was available in the YRL that could potentially have served as a replacement.

\noindent\textbf{4. Significant, un-addressed bugs or issues in the software.}

Bugs are a risk inherent in all software development, and  can cause `finished' software to be unstable or functionally flawed if undetected or left unaddressed. In order to mitigate the chances of a major bug going undetected a thorough testing phase was included in the project plan. During this time testing methodologies were applied to the software in an attempt to detect and identify any bugs in the software. Any bugs discovered could then be fixed, or if a fix was not possible, mitigated in some way. The testing applied was based on established, proven methodologies, in an effort to ensure that bugs did not go undetected due to a poor testing strategy or poor test coverage.

%----------------------------------------------------------------------------------------

\section{Application of Agile Methodologies} \label{Agile}
`\textit{Agile Software Development}' \cite{Agile} is the term used to describe a variety of software development principles and practices, which focus on the continuous, incremental delivery of functional software, and the ability to rapidly adapt to changing requirements, among other things. A number of key principles from the agile development methodology were embraced during this project. The first was the idea that functional, working software should always be a priority, and should be the primary measure of progress. The software for this project was therefore developed incrementally, and was verified to still be in a usable state after the addition of each new element or feature. This also tied into the MVP approach discussed previously, as after each incremental addition the software could undergo a cursory evaluation to determine if it yet satisfied the stated aims. 

The second agile development principle that guided this project was the idea that reacting to changes in requirements or changes in the feasibility of features is more important than strict adherence to a plan. Whenever an issue arose that changed the estimated time required to implement a feature, or brought into question its feasibility, the necessity of implementing that feature was reconsidered in light of the new information. Features that were determined to be likely to require too much time when compared to their value were discarded in favour of other key features that could be realised more quickly. In practice relatively few issues arose during the development of this project, and most features were able to implemented as planned, within the original estimated time frames.

One further principle from the agile development methodology that also had an effect on the project was the principle of customer collaboration and continual requirement capture. Although this project did not have a customer in the traditional sense, the researchers within the YRL who responded to an initial survey regarding the software acted as potential customers, guiding the design and implementation of the system with their input. The priority of certain features were adjusted following their comments, with the hope that this would lead to a more useful and practical implementation of the system.

%----------------------------------------------------------------------------------------

\section{Summary}

This chapter has presented the plan that was devised early in the project for completing the necessary work in a timely fashion. The plan featured three main phases; an analysis and design phase, an implementation phase, and a testing and evaluation phase. Each phase was then broken down into individual tasks necessary to satisfy the project requirements and functional specification, and a Gantt chart was generated based on the estimated time requirements of each task. Potential risks to the project were also presented and analysed. Hardware malfunction or loss was identified as a significant existential risk, however a number of mitigation techniques were employed, including redundancy hardware where possible. The other key risks - insufficient time, and bugs in the finished software - were to be mitigated through planning and adaptive scheduling, and thorough testing respectively. Finally a summary of the application of elements from the \textit{Agile} software development methodology to this project was given, including a focus on functional software and incremental development, reactive planning, and frequent requirements capture.