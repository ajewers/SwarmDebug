% Chapter 1

\chapter[Introduction]{Introduction} % Main chapter title

\label{Chapter1} % For referencing the chapter elsewhere, use \ref{Chapter1} 

%----------------------------------------------------------------------------------------

% Define some commands to keep the formatting separated from the content 
\newcommand{\keyword}[1]{\textbf{#1}}
\newcommand{\tabhead}[1]{\textbf{#1}}
\newcommand{\code}[1]{\texttt{#1}}
\newcommand{\file}[1]{\texttt{\bfseries#1}}
\newcommand{\option}[1]{\texttt{\itshape#1}}

%----------------------------------------------------------------------------------------

\section{Overview}
Recent years have seen rapid development in the use of robotics to perform increasingly complex tasks and solve ever more complex problems, due the constantly increasing availability of computing power and reductions in the cost of hardware such as digital sensors and actuators, as well as developments in the application of forms of artificial intelligence to robot control. Many new areas of robotics research have emerged, and new robotic paradigms have been created.

Swarm robotics has emerged as a promising area of research for solving problems which would be infeasibly difficult or expensive for a conventional robotics approach. Studies of the self-organising behaviour of social insect colonies, and the use of these studies to develop mathematical models for using a large number of individual agents solve problems collectively, based on local information, led to the development of a field referred to as Swarm Intelligence (SI). Swarm Robotics then developed from a desire to apply these concepts to 

%----------------------------------------------------------------------------------------

\section{Project Concept}
Debugging swarm robotics is hard. Real time real world systems present issues. Can we make it easier for people by developing this piece of software?
