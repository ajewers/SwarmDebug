% Chapter 1

\chapter[Introduction]{Introduction} % Main chapter title

\label{Chapter1} % For referencing the chapter elsewhere, use \ref{Chapter1} 

%----------------------------------------------------------------------------------------

% Define some commands to keep the formatting separated from the content 
\newcommand{\keyword}[1]{\textbf{#1}}
\newcommand{\tabhead}[1]{\textbf{#1}}
\newcommand{\code}[1]{\texttt{#1}}
\newcommand{\file}[1]{\texttt{\bfseries#1}}
\newcommand{\option}[1]{\texttt{\itshape#1}}

%----------------------------------------------------------------------------------------

\section{Overview}
Recent years have seen rapid development in robotics technology due the constantly increasing availability of computing power, reductions in the cost of hardware such as digital sensors and actuators, and developments in the application of artificial intelligence to robot control. This has lead to robots being used to perform increasingly complex tasks and solve ever more complex problems. Many new areas of robotics research have emerged as a result as researchers strive to find new and better ways to apply this technology, entering into problem domains once thought to be solely the domain of humans. Whole new robotics paradigms have been created as the standard model of a single, very complex, very expensive robot has been questioned, opening the door for cooperative robots, multi-robot systems, and more specifically swarm robotics.

Studies into the self-organising behaviour of social insect colonies, and the development of mathematical models based on these behaviours  led to the development of a field referred to as Swarm Intelligence (SI). The aim of these models is to use a large number of individual agents to solve problems collectively, with each agent using only local information, and without any centralised control. Swarm Robotics developed from a desire to apply these concepts in practice to real world problem solving. Swarm robotics has since emerged as a promising area of research for solving problems which would be infeasibly difficult or expensive for a conventional robotics approach.

%----------------------------------------------------------------------------------------

\section{Project Concept}
Debugging swarm robotics is hard. Real time real world systems present issues. Can we make it easier for people by developing this piece of software?
