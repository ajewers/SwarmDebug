% Chapter 10

\chapter[Testing and Evaluation]{Testing and Evaluation} % Main chapter title

\label{Chapter10} % For referencing the chapter elsewhere, use \ref{Chapter10} 

%----------------------------------------------------------------------------------------

Once the implementation phase of the project was complete it was important to thoroughly test the software in order to verify its correct implementation and operation, and identify any remaining issues that needed to be fixed or mitigated. Some testing was also carried out during the implementation phase in order to verify the correct operation of individual components as they were completed, and to ensure that different components would work correctly together. This is referred to here as continuous integration testing. This was done to reduce the chances of issues stacking up and becoming layered or entrenched as development continued. A more rigorous testing process was then applied once the implementation was complete, and included testing of both the user interface and the system back end, as well as testing the system as a whole. Once these testing processes had been completed, and any issues addressed, an evaluation process was undertaken to determine whether the system was useful in practice, and to gauge what benefits it provided the user, whether it succeeded in meeting the aims of the project, and how it might be improved in the future. The evaluation process involved carrying out observed trials with a number of potential users of the system, as well as a questionnaire completed by participants after having used the system. This section gives details of the different testing stages and the evaluation process.

%----------------------------------------------------------------------------------------

\section{Continuous Integration Testing} \label{ContinuousIntegrationTesting}


%----------------------------------------------------------------------------------------

\section{Manual User Interface Testing} \label{ManualUserInterfaceTesting}
The purpose of any graphical user interface is to present information to a human user and collect their input. It is therefore important that user interfaces be tested manually by a human user, as automated testing methods are often not sufficient for or not capable of verifying that information is displayed legibly and correctly, and that user input functions properly. The user interface of the system is one of the most important parts of the project, and therefore a thorough manual user interface testing process was undertaken.

\subsection{Method}
The general approach taken to the user interface testing can be summarised as follows:

\begin{enumerate}
 \item Separate UI into individual elements.
 \item State the purpose and required functionality of each elements.
 \item Define a general test strategy for a single UI element as a series of checks.
 \item Identify any special case components, and define different test strategies where necessary.
 \item Apply the relevant strategy to each interface element in turn.
\end{enumerate}

\vspace{0.5cm}

The user interface was separated into the elements described in table \ref{tab:UserInterfaceElements}. Special case elements requiring different test strategies are highlighted in bold.

\begin{longtable}{ l c }
\caption[User Interface Elements]{Individual user interface elements that require testing.}\\
 Element & Test Strategy\\
 \hline
 Visualiser Panel Tab System & A \\
 \textbf{Visualiser} & \textbf{B} \\
 Visualiser Settings Tab & A \\
 Camera Settings Tab & A \\
 Robot List Panel Tab System & A \\
 Robot List Element & A \\
 Network Settings Tab & A \\
 Logging Settings Tab & A \\
 Data Panel Tab System & A \\
 Console Data Tab & A \\
 Overview Data Tab & A \\
 State Data Tab & A \\
 IR Data Tab & A \\
 Custom Data Tab & A \\
 Individual Visulisation Settings Dialogs & A \\
 \bottomrule
 \label{tab:UserInterfaceElements}
\end{longtable}

The following test strategies were then devised for the different element categories.

\textbf{Test Strategy A - Standard UI Elements:}

\begin{enumerate}
 \item Examine UI element visually. Verify that it appears correct. Verify that it contains all elements necessary to satisfy its purpose.
 \item Examine all text within the element. Check for errors in both meaning and spelling.
 \item Verify that all components within the element which perform actions in response to user input operate correctly.
 \item Verify that all components respond quickly to user input.
 \item Verify that component actions and functionality do not degrade with extreme use (sustained rapid input, large numbers of input changes, etc).
 \item Verify that all data displayed within the element is visible, readable, correctly arranged and correctly labelled.
 \item Verify that the element behaves sensibly when window resizing occurs, and that it remains usable and data remains visible whenever possible.
 \item Verify that the element updates promptly when responding to changes in data.
\end{enumerate}

\textbf{Test Strategy B - Visualiser:}

\begin{enumerate}
 \item Verify that the video image is displayed correctly.
 \item Verify that user clicks within the visualiser space are located correctly, at a number of different window sizes.
 \item Verify that robots can be selected by clicking on their location in the visualiser image.
 \item For each data visualisation type:
 \begin{enumerate}
  \item Define a set of input data and the expected representation of this data.
  \item Supply the input data.
  \item Verify the representation is as expected.
  \item Check that the visualisation is clear and any text is legible.
  \item Repeat for multiple sets of input data.
  \item Verify that integrity is maintained with extreme data, corner cases and zero data, wherever possible.
  \item Verify that integrity is maintained at a range of window sizes, within reasonable limits.
 \end{enumerate}
\end{enumerate}


\subsection{Results}
For each user interface element the appropriate test strategy, as specified in table \ref{tab:UserInterfaceElements}, was carried out. The following results were obtained.

\begin{longtable}{ l p{10cm} }
 \hline
 \multicolumn{2}{c}{\textbf{Visualiser Panel Tab System}}\\
 \hline
 \textbf{Purpose} & Allows access to the different tabs within the visualiser panel.\\
 \textbf{Required Functionality} & Must display the names of each different tab. Must allow the user to click on the tabs and thus change between them. The required tabs are the visualiser tab, the visualiser settings tab and the tracking camera settings tab.\\
 \hline
 \multicolumn{2}{p{14cm}}{\textbf{1. Examine UI element visually. Verify that it appears correct. Verify that it contains all elements necessary to satisfy its purpose.}}\\
 \multicolumn{2}{p{14cm}}{The tab bar appears to be visually correct. All required tabs are present and visible.}\\
 \hline
 \multicolumn{2}{p{14cm}}{\textbf{2. Examine all text within the element. Check for errors in both meaning and spelling.}}\\
 \multicolumn{2}{p{14cm}}{The spelling of each tab name is correct. The meaning of each tab name is relatively clear, although 'Settings' is ambiguous, and only related to the visualiser through position and context. Consider renaming this tab to clarify its purpose.}\\
 \hline
 \multicolumn{2}{p{14cm}}{\textbf{3. Verify that all components within the element which perform actions in response to user input operate correctly.}}\\
 \multicolumn{2}{p{14cm}}{Tab selection operates correctly. Clicking any of the tabs changes the panel to display the contents of that tab. This satisfies the required functionality.}\\
 \hline
 \multicolumn{2}{p{14cm}}{\textbf{4. Verify that all components respond quickly to user input.}}\\
 \multicolumn{2}{p{14cm}}{The tab changes immediately when clicked.}\\
 \hline
 \multicolumn{2}{p{14cm}}{\textbf{5. Verify that component actions and functionality do not degrade with extreme use (sustained rapid input, large numbers of input changes, etc).}}\\
 \multicolumn{2}{p{14cm}}{Repeatedly and rapidly changing tabs does not have any detrimental affect on the application.}\\
 \hline
 \multicolumn{2}{p{14cm}}{\textbf{6. Verify that all data displayed within the element is visible, readable, correctly arranged and correctly labelled.}}\\
 \multicolumn{2}{p{14cm}}{n/a.}\\
 \hline
 \multicolumn{2}{p{14cm}}{\textbf{7. Verify that the element behaves sensibly when window resizing occurs, and that it remains usable and data remains visible whenever possible.}}\\
 \multicolumn{2}{p{14cm}}{The tabs are not affected by window resizing. All of the tabs fit within the minimum size of the panel. If the panel is reduced below this size it is minimized, and the tabs are hidden as intended.}\\
 \hline
 \multicolumn{2}{p{14cm}}{\textbf{8. Verify that the element updates promptly when responding to changes in data.}}\\
 \multicolumn{2}{p{14cm}}{n/a.}\\
 \hline
 \textbf{Fixes Required} & Consider changing the text on the visualiser settings tab from 'Settings' to something more informative to improve clarity.\\
 \bottomrule
\end{longtable}
\clearpage

\begin{longtable}{ l p{10cm} }
 \hline
 \multicolumn{2}{c}{\textbf{Visualiser Settings Tab}}\\
 \hline
 \textbf{Purpose} & Displays the current settings for the visualiser and allows the user to change them.\\
 \textbf{Required Functionality} & Must correctly display the current visualiser settings. Must allow the user to adjust the general visualiser settings, and access dialog windows for changing the settings of specific visualisations.\\
 \hline
 \multicolumn{2}{p{14cm}}{\textbf{1. Examine UI element visually. Verify that it appears correct. Verify that it contains all elements necessary to satisfy its purpose.}}\\
 \multicolumn{2}{p{14cm}}{Panel layout appears to be correct. Controls are included to modify all of the required settings.}\\
 \hline
 \multicolumn{2}{p{14cm}}{\textbf{2. Examine all text within the element. Check for errors in both meaning and spelling.}}\\
 \multicolumn{2}{p{14cm}}{All spelling is correct, and the meaning of all text is clear.}\\
 \hline
 \multicolumn{2}{p{14cm}}{\textbf{3. Verify that all components within the element which perform actions in response to user input operate correctly.}}\\
 \multicolumn{2}{p{14cm}}{All the settings controls work correctly. Double clicking on any of the visualiser config elements in the list displays the appropriate settings dialog, if one exists.}\\
 \hline
 \multicolumn{2}{p{14cm}}{\textbf{4. Verify that all components respond quickly to user input.}}\\
 \multicolumn{2}{p{14cm}}{All controls respond immediately to input.}\\
 \hline
 \multicolumn{2}{p{14cm}}{\textbf{5. Verify that component actions and functionality do not degrade with extreme use (sustained rapid input, large numbers of input changes, etc).}}\\
 \multicolumn{2}{p{14cm}}{Rapidly enabling and disabling settings multiple times has no detrimental affect. Disabling and re-enabling the robot colours setting causes the robots to be re-assigned colours randomly, which might be undesired behaviour.}\\
 \hline
 \multicolumn{2}{p{14cm}}{\textbf{6. Verify that all data displayed within the element is visible, readable, correctly arranged and correctly labelled.}}\\
 \multicolumn{2}{p{14cm}}{Information regarding the current settings is readable, correctly arranged and clearly labelled. Detailed settings for each visualiser element are described in text next to the element name.}\\
 \hline
 \multicolumn{2}{p{14cm}}{\textbf{7. Verify that the element behaves sensibly when window resizing occurs, and that it remains usable and data remains visible whenever possible.}}\\
 \multicolumn{2}{p{14cm}}{All text fits within the minimum panel width, and scroll bars are presented when the height becomes too small to display the full visualiser config element list. Resizing is handled gracefully.}\\
 \hline
 \multicolumn{2}{p{14cm}}{\textbf{8. Verify that the element updates promptly when responding to changes in data.}}\\
 \multicolumn{2}{p{14cm}}{Changes to the general settings and the visualiser settings are reflected in the panel immediately.}\\
 \hline
 \textbf{Fixes Required} & Investigate alternative methods for assigning robot colours.\\
 \bottomrule
\end{longtable}
\clearpage

\begin{longtable}{ l p{10cm} }
 \hline
 \multicolumn{2}{c}{\textbf{Camera Settings Tab}}\\
 \hline
 \textbf{Purpose} & Displays the current settings for the tracking camera and allows the user to change them.\\
 \textbf{Required Functionality} & Must correctly display the current camera settings. Must allow the user to adjust the camera settings. Must allow the user to adjust the tracking system settings, and apply/remove mappings between specific tracking tag IDs and robot IDs.\\
 \hline
 \multicolumn{2}{p{14cm}}{\textbf{1. Examine UI element visually. Verify that it appears correct. Verify that it contains all elements necessary to satisfy its purpose.}}\\
 \multicolumn{2}{p{14cm}}{Panel layout appears to be correct. Controls are included to modify all of the required settings. A table and associated controls are included to allow the ID mapping to be modified.}\\
 \hline
 \multicolumn{2}{p{14cm}}{\textbf{2. Examine all text within the element. Check for errors in both meaning and spelling.}}\\
 \multicolumn{2}{p{14cm}}{All spelling is correct, and the meaning of all text is clear.}\\
 \hline
 \multicolumn{2}{p{14cm}}{\textbf{3. Verify that all components within the element which perform actions in response to user input operate correctly.}}\\
 \multicolumn{2}{p{14cm}}{All the settings controls and the mapping table work correctly.}\\
 \hline
 \multicolumn{2}{p{14cm}}{\textbf{4. Verify that all components respond quickly to user input.}}\\
 \multicolumn{2}{p{14cm}}{All controls respond immediately to input.}\\
 \hline
 \multicolumn{2}{p{14cm}}{\textbf{5. Verify that component actions and functionality do not degrade with extreme use (sustained rapid input, large numbers of input changes, etc).}}\\
 \multicolumn{2}{p{14cm}}{Camera resolution input boxes are input validated to only accept numbers in the range 1 to 10,000. Rapid use of controls has no detrimental affect.}\\
 \hline
 \multicolumn{2}{p{14cm}}{\textbf{6. Verify that all data displayed within the element is visible, readable, correctly arranged and correctly labelled.}}\\
 \multicolumn{2}{p{14cm}}{Current settings are all correctly displayed.}\\
 \hline
 \multicolumn{2}{p{14cm}}{\textbf{7. Verify that the element behaves sensibly when window resizing occurs, and that it remains usable and data remains visible whenever possible.}}\\
 \multicolumn{2}{p{14cm}}{All components fit within the panel's minimum dimensions, and are hidden when the panel is minimized.}\\
 \hline
 \multicolumn{2}{p{14cm}}{\textbf{8. Verify that the element updates promptly when responding to changes in data.}}\\
 \multicolumn{2}{p{14cm}}{n/a.}\\
 \hline
 \textbf{Fixes Required} & None.\\
 \bottomrule
\end{longtable}
\clearpage

\begin{longtable}{ l p{10cm} }
\hline
 \multicolumn{2}{c}{\textbf{Robot List Panel Tab System}}\\
 \hline
 \textbf{Purpose} & Allows access to the different tabs within the robot list panel.\\
 \textbf{Required Functionality} & Must display the names of each different tab. Must allow the user to click on the tabs and thus change between them. The required tabs are the robot list tab, the network settings tab and the logging tab.\\
 \hline
 \multicolumn{2}{p{14cm}}{\textbf{1. Examine UI element visually. Verify that it appears correct. Verify that it contains all elements necessary to satisfy its purpose.}}\\
 \multicolumn{2}{p{14cm}}{The tab bar appears to be visually correct. All required tabs are present and visible.}\\
 \hline
 \multicolumn{2}{p{14cm}}{\textbf{2. Examine all text within the element. Check for errors in both meaning and spelling.}}\\
 \multicolumn{2}{p{14cm}}{The spelling of each tab name is correct. The meaning of each tab name is clear.}\\
 \hline
 \multicolumn{2}{p{14cm}}{\textbf{3. Verify that all components within the element which perform actions in response to user input operate correctly.}}\\
 \multicolumn{2}{p{14cm}}{Tab selection operates correctly. Clicking any of the tabs changes the panel to display the contents of that tab. This satisfies the required functionality.}\\
 \hline
 \multicolumn{2}{p{14cm}}{\textbf{4. Verify that all components respond quickly to user input.}}\\
 \multicolumn{2}{p{14cm}}{The tab changes immediately when clicked.}\\
 \hline
 \multicolumn{2}{p{14cm}}{\textbf{5. Verify that component actions and functionality do not degrade with extreme use (sustained rapid input, large numbers of input changes, etc).}}\\
 \multicolumn{2}{p{14cm}}{Repeatedly and rapidly changing tabs does not have any detrimental affect on the application.}\\
 \hline
 \multicolumn{2}{p{14cm}}{\textbf{6. Verify that all data displayed within the element is visible, readable, correctly arranged and correctly labelled.}}\\
 \multicolumn{2}{p{14cm}}{n/a.}\\
 \hline
 \multicolumn{2}{p{14cm}}{\textbf{7. Verify that the element behaves sensibly when window resizing occurs, and that it remains usable and data remains visible whenever possible.}}\\
 \multicolumn{2}{p{14cm}}{When the panel width is reduced such that the tabs do not fit, small scroll arrows are added to allow the user to scroll along the tabs, therefore remaining accessible even when the window or panel size is reduced.}\\
 \hline
 \multicolumn{2}{p{14cm}}{\textbf{8. Verify that the element updates promptly when responding to changes in data.}}\\
 \multicolumn{2}{p{14cm}}{n/a.}\\
 \hline
 \textbf{Fixes Required} & None.\\
 \bottomrule
\end{longtable}
\clearpage

\begin{longtable}{ l p{10cm} }
 \hline
 \multicolumn{2}{c}{\textbf{Robot List}}\\
 \hline
 \textbf{Purpose} & Displays a list of all the robots for which the system has data, allowing the user to select them.\\
 \textbf{Required Functionality} & Must display a list of all the known robots, identifying them by both ID and name. Must allow the user to select any of the robots, causing the other parts of the interface to target the newly selected robot.\\
 \hline
 \multicolumn{2}{p{14cm}}{\textbf{1. Examine UI element visually. Verify that it appears correct. Verify that it contains all elements necessary to satisfy its purpose.}}\\
 \multicolumn{2}{p{14cm}}{The list displays correctly. Contains all know robots. Clearly shows current selection.}\\
 \hline
 \multicolumn{2}{p{14cm}}{\textbf{2. Examine all text within the element. Check for errors in both meaning and spelling.}}\\
 \multicolumn{2}{p{14cm}}{n/a.}\\
 \hline
 \multicolumn{2}{p{14cm}}{\textbf{3. Verify that all components within the element which perform actions in response to user input operate correctly.}}\\
 \multicolumn{2}{p{14cm}}{Robots can be selected correctly by clicking, and changing the selection causes the rest of the application to update to the new focus.}\\
 \hline
 \multicolumn{2}{p{14cm}}{\textbf{4. Verify that all components respond quickly to user input.}}\\
 \multicolumn{2}{p{14cm}}{The application responds to a new selection immediately.}\\
 \hline
 \multicolumn{2}{p{14cm}}{\textbf{5. Verify that component actions and functionality do not degrade with extreme use (sustained rapid input, large numbers of input changes, etc).}}\\
 \multicolumn{2}{p{14cm}}{Rapidly changing the selected robot has no detrimental effect on the application.}\\
 \hline
 \multicolumn{2}{p{14cm}}{\textbf{6. Verify that all data displayed within the element is visible, readable, correctly arranged and correctly labelled.}}\\
 \multicolumn{2}{p{14cm}}{For each robot the correct ID number and name are displayed, as defined in the data model. The the number of robots exceeds the available space in the list a scroll bar is added.}\\
 \hline
 \multicolumn{2}{p{14cm}}{\textbf{7. Verify that the element behaves sensibly when window resizing occurs, and that it remains usable and data remains visible whenever possible.}}\\
 \multicolumn{2}{p{14cm}}{The list is unaffected by window resizing, adding a scroll bar when necessary at small sizes.}\\
 \hline
 \multicolumn{2}{p{14cm}}{\textbf{8. Verify that the element updates promptly when responding to changes in data.}}\\
 \multicolumn{2}{p{14cm}}{Changes in the data model are reflected immediately, including robots being added, removed and changing name.}\\
 \hline
 \textbf{Fixes Required} & None.\\
 \bottomrule
\end{longtable}
\clearpage

\begin{longtable}{ l p{10cm} }
 \hline
 \multicolumn{2}{c}{\textbf{Network Settings Tab}}\\
 \hline
 \textbf{Purpose} & Allows the user to configure settings related to the network communication with the robots.\\
 \textbf{Required Functionality} & Must display the current network settings. Must allow the user to change the network settings. Must allow the user to start and stop the data thread which listens for packets.\\
 \hline
 \multicolumn{2}{p{14cm}}{\textbf{1. Examine UI element visually. Verify that it appears correct. Verify that it contains all elements necessary to satisfy its purpose.}}\\
 \multicolumn{2}{p{14cm}}{The panel appears visually correct, with the necessary components visible.}\\
 \hline
 \multicolumn{2}{p{14cm}}{\textbf{2. Examine all text within the element. Check for errors in both meaning and spelling.}}\\
 \multicolumn{2}{p{14cm}}{All spelling correct and all meanings clear.}\\
 \hline
 \multicolumn{2}{p{14cm}}{\textbf{3. Verify that all components within the element which perform actions in response to user input operate correctly.}}\\
 \multicolumn{2}{p{14cm}}{The port number can be changed correctly, and is input-validated to reject non-numerical input. The button for starting and stopping the data thread operates correctly.}\\
 \hline
 \multicolumn{2}{p{14cm}}{\textbf{4. Verify that all components respond quickly to user input.}}\\
 \multicolumn{2}{p{14cm}}{All components respond immediately.}\\
 \hline
 \multicolumn{2}{p{14cm}}{\textbf{5. Verify that component actions and functionality do not degrade with extreme use (sustained rapid input, large numbers of input changes, etc).}}\\
 \multicolumn{2}{p{14cm}}{Rapidly and repeatedly pressing the start/stop listening button appears to have no detrimental effect. The port number entry box rejects numbers below 1, but does not have a maximum value, as some computers support extremely high port numbers.}\\
 \hline
 \multicolumn{2}{p{14cm}}{\textbf{6. Verify that all data displayed within the element is visible, readable, correctly arranged and correctly labelled.}}\\
 \multicolumn{2}{p{14cm}}{n/a.}\\
 \hline
 \multicolumn{2}{p{14cm}}{\textbf{7. Verify that the element behaves sensibly when window resizing occurs, and that it remains usable and data remains visible whenever possible.}}\\
 \multicolumn{2}{p{14cm}}{Controls remain usable at the full range of interface sizes.}\\
 \hline
 \multicolumn{2}{p{14cm}}{\textbf{8. Verify that the element updates promptly when responding to changes in data.}}\\
 \multicolumn{2}{p{14cm}}{n/a.}\\
 \hline
 \textbf{Fixes Required} & None.\\
 \bottomrule
\end{longtable}
\clearpage

\begin{longtable}{ l p{10cm} }
 \hline
 \multicolumn{2}{c}{\textbf{Data Logging Tab}}\\
 \hline
 \textbf{Purpose} & Allows the user to configure the data logging functionality.\\
 \textbf{Required Functionality} & Must display the current data logging settings. Must allow the user to change the data logging settings. Must allow the user to start and stop the data logging.\\
 \hline
 \multicolumn{2}{p{14cm}}{\textbf{1. Examine UI element visually. Verify that it appears correct. Verify that it contains all elements necessary to satisfy its purpose.}}\\
 \multicolumn{2}{p{14cm}}{The panel appears correct, with all required elements present.}\\
 \hline
 \multicolumn{2}{p{14cm}}{\textbf{2. Examine all text within the element. Check for errors in both meaning and spelling.}}\\
 \multicolumn{2}{p{14cm}}{Spellings correct and meanings clear. The log file directory path can be extremely long, and might overrun the available width of the panel. Long paths should be truncated.}\\
 \hline
 \multicolumn{2}{p{14cm}}{\textbf{3. Verify that all components within the element which perform actions in response to user input operate correctly.}}\\
 \multicolumn{2}{p{14cm}}{All components operate correctly. The log file path can be set using a file dialog.}\\
 \hline
 \multicolumn{2}{p{14cm}}{\textbf{4. Verify that all components respond quickly to user input.}}\\
 \multicolumn{2}{p{14cm}}{All components respond immediately.}\\
 \hline
 \multicolumn{2}{p{14cm}}{\textbf{5. Verify that component actions and functionality do not degrade with extreme use (sustained rapid input, large numbers of input changes, etc).}}\\
 \multicolumn{2}{p{14cm}}{Rapid use of the start/stop button has no detrimental effect. Invalid file selections are rejected by the dialog.}\\
 \hline
 \multicolumn{2}{p{14cm}}{\textbf{6. Verify that all data displayed within the element is visible, readable, correctly arranged and correctly labelled.}}\\
 \multicolumn{2}{p{14cm}}{n/a.}\\
 \hline
 \multicolumn{2}{p{14cm}}{\textbf{7. Verify that the element behaves sensibly when window resizing occurs, and that it remains usable and data remains visible whenever possible.}}\\
 \multicolumn{2}{p{14cm}}{All controls fit within the minimum size of the panel. See 2. for note on the log file path length.}\\
 \hline
 \multicolumn{2}{p{14cm}}{\textbf{8. Verify that the element updates promptly when responding to changes in data.}}\\
 \multicolumn{2}{p{14cm}}{n/a.}\\
 \hline
 \textbf{Fixes Required} & Add code to handle long directory paths gracefully.\\
 \bottomrule
\end{longtable}
\clearpage

\begin{longtable}{ l p{10cm} }
\hline
 \multicolumn{2}{c}{\textbf{Data Panel Tab System}}\\
 \hline
 \textbf{Purpose} & Allows access to the different tabs within the data panel.\\
 \textbf{Required Functionality} & Must display the names of each different tab. Must allow the user to click on the tabs and thus change between them. The required tabs are: Console, Overview, State, IR data, Custom data.\\
 \hline
 \multicolumn{2}{p{14cm}}{\textbf{1. Examine UI element visually. Verify that it appears correct. Verify that it contains all elements necessary to satisfy its purpose.}}\\
 \multicolumn{2}{p{14cm}}{The tab bar appears to be visually correct. All required tabs are present and visible.}\\
 \hline
 \multicolumn{2}{p{14cm}}{\textbf{2. Examine all text within the element. Check for errors in both meaning and spelling.}}\\
 \multicolumn{2}{p{14cm}}{The spelling of each tab name is correct. The meaning of each tab name is clear.}\\
 \hline
 \multicolumn{2}{p{14cm}}{\textbf{3. Verify that all components within the element which perform actions in response to user input operate correctly.}}\\
 \multicolumn{2}{p{14cm}}{Tab selection operates correctly. Clicking any of the tabs changes the panel to display the contents of that tab. This satisfies the required functionality.}\\
 \hline
 \multicolumn{2}{p{14cm}}{\textbf{4. Verify that all components respond quickly to user input.}}\\
 \multicolumn{2}{p{14cm}}{The tab changes immediately when clicked.}\\
 \hline
 \multicolumn{2}{p{14cm}}{\textbf{5. Verify that component actions and functionality do not degrade with extreme use (sustained rapid input, large numbers of input changes, etc).}}\\
 \multicolumn{2}{p{14cm}}{Repeatedly and rapidly changing tabs does not have any detrimental affect on the application.}\\
 \hline
 \multicolumn{2}{p{14cm}}{\textbf{6. Verify that all data displayed within the element is visible, readable, correctly arranged and correctly labelled.}}\\
 \multicolumn{2}{p{14cm}}{n/a.}\\
 \hline
 \multicolumn{2}{p{14cm}}{\textbf{7. Verify that the element behaves sensibly when window resizing occurs, and that it remains usable and data remains visible whenever possible.}}\\
 \multicolumn{2}{p{14cm}}{All tabs fit within the minimum window width.}\\
 \hline
 \multicolumn{2}{p{14cm}}{\textbf{8. Verify that the element updates promptly when responding to changes in data.}}\\
 \multicolumn{2}{p{14cm}}{n/a.}\\
 \hline
 \textbf{Fixes Required} & None.\\
 \bottomrule
\end{longtable}
\clearpage

\begin{longtable}{ l p{10cm} }
 \hline
 \multicolumn{2}{c}{\textbf{Console Tab}}\\
 \hline
 \textbf{Purpose} & Displays a text based console which reports messages regarding application events and messages from the robots.\\
 \textbf{Required Functionality} & Must contain a text console. Must display messages regarding the application and messages from the robots. Must display messages in order. Must identify the source of all robot messages. Must update immediately when a message is received from either source. \\
 \hline
 \multicolumn{2}{p{14cm}}{\textbf{1. Examine UI element visually. Verify that it appears correct. Verify that it contains all elements necessary to satisfy its purpose.}}\\
 \multicolumn{2}{p{14cm}}{The console is visually correct, however the most recent message is displayed on the top line, counter-intuitively.}\\
 \hline
 \multicolumn{2}{p{14cm}}{\textbf{2. Examine all text within the element. Check for errors in both meaning and spelling.}}\\
 \multicolumn{2}{p{14cm}}{The text within the element comes from the messages, hence cannot be checked here.}\\
 \hline
 \multicolumn{2}{p{14cm}}{\textbf{3. Verify that all components within the element which perform actions in response to user input operate correctly.}}\\
 \multicolumn{2}{p{14cm}}{Messages are presented correctly and in order. Robot messages are identified correctly.}\\
 \hline
 \multicolumn{2}{p{14cm}}{\textbf{4. Verify that all components respond quickly to user input.}}\\
 \multicolumn{2}{p{14cm}}{n/a.}\\
 \hline
 \multicolumn{2}{p{14cm}}{\textbf{5. Verify that component actions and functionality do not degrade with extreme use (sustained rapid input, large numbers of input changes, etc).}}\\
 \multicolumn{2}{p{14cm}}{Long messages and large numbers of messages are handled gracefully, and remain readable using the automatic scroll bars.}\\
 \hline
 \multicolumn{2}{p{14cm}}{\textbf{6. Verify that all data displayed within the element is visible, readable, correctly arranged and correctly labelled.}}\\
 \multicolumn{2}{p{14cm}}{All messages are readable and clearly labelled.}\\
 \hline
 \multicolumn{2}{p{14cm}}{\textbf{7. Verify that the element behaves sensibly when window resizing occurs, and that it remains usable and data remains visible whenever possible.}}\\
 \multicolumn{2}{p{14cm}}{Resizing teh window has no detrimental effect on the console.}\\
 \hline
 \multicolumn{2}{p{14cm}}{\textbf{8. Verify that the element updates promptly when responding to changes in data.}}\\
 \multicolumn{2}{p{14cm}}{New messages appear immediately.}\\
 \hline
 \textbf{Fixes Required} & Adjust the ordering so that the most recent message appears on the bottom line, as is the standard for text consoles.\\
 \bottomrule
\end{longtable}
\clearpage

\begin{longtable}{ l p{10cm} }
 \hline
 \multicolumn{2}{c}{\textbf{Overview Tab}}\\
 \hline
 \textbf{Purpose} & Displays a summary of the data related to the selected robot.\\
 \textbf{Required Functionality} & Must display data related to the selected robot, including ID, name, state and position/orientation values. Must update immediately when new data is received.\\
 \hline
 \multicolumn{2}{p{14cm}}{\textbf{1. Examine UI element visually. Verify that it appears correct. Verify that it contains all elements necessary to satisfy its purpose.}}\\
 \multicolumn{2}{p{14cm}}{The tab appears visually correct. The heading text appears redundant as the tab itself already contains the heading.}\\
 \hline
 \multicolumn{2}{p{14cm}}{\textbf{2. Examine all text within the element. Check for errors in both meaning and spelling.}}\\
 \multicolumn{2}{p{14cm}}{All spelling correct and all meanings clear.}\\
 \hline
 \multicolumn{2}{p{14cm}}{\textbf{3. Verify that all components within the element which perform actions in response to user input operate correctly.}}\\
 \multicolumn{2}{p{14cm}}{n/a.}\\
 \hline
 \multicolumn{2}{p{14cm}}{\textbf{4. Verify that all components respond quickly to user input.}}\\
 \multicolumn{2}{p{14cm}}{n/a.}\\
 \hline
 \multicolumn{2}{p{14cm}}{\textbf{5. Verify that component actions and functionality do not degrade with extreme use (sustained rapid input, large numbers of input changes, etc).}}\\
 \multicolumn{2}{p{14cm}}{n/a.}\\
 \hline
 \multicolumn{2}{p{14cm}}{\textbf{6. Verify that all data displayed within the element is visible, readable, correctly arranged and correctly labelled.}}\\
 \multicolumn{2}{p{14cm}}{All of the data points are visible, readable, clear, correctly arranged and labelled.}\\
 \hline
 \multicolumn{2}{p{14cm}}{\textbf{7. Verify that the element behaves sensibly when window resizing occurs, and that it remains usable and data remains visible whenever possible.}}\\
 \multicolumn{2}{p{14cm}}{All data is visible at the full range of window size.}\\
 \hline
 \multicolumn{2}{p{14cm}}{\textbf{8. Verify that the element updates promptly when responding to changes in data.}}\\
 \multicolumn{2}{p{14cm}}{Updates to the data are reflected immediately.}\\
 \hline
 \textbf{Fixes Required} & Remove the overview heading from inside the tab, as it is redundant.\\
 \bottomrule
\end{longtable}
\clearpage

\begin{longtable}{ l p{10cm} }
 \hline
 \multicolumn{2}{c}{\textbf{State Tab}}\\
 \hline
 \textbf{Purpose} & Displays information related to the internal state of the robot.\\
 \textbf{Required Functionality} & Must display a list of the selected robot's known states. Must display a list of the selected robots recent state changes, including timing information. Must update immediately when a state change occurs.\\
 \hline
 \multicolumn{2}{p{14cm}}{\textbf{1. Examine UI element visually. Verify that it appears correct. Verify that it contains all elements necessary to satisfy its purpose.}}\\
 \multicolumn{2}{p{14cm}}{Appears visually correct. Both lists are present.}\\
 \hline
 \multicolumn{2}{p{14cm}}{\textbf{2. Examine all text within the element. Check for errors in both meaning and spelling.}}\\
 \multicolumn{2}{p{14cm}}{All spelling correct and meanings clear.}\\
 \hline
 \multicolumn{2}{p{14cm}}{\textbf{3. Verify that all components within the element which perform actions in response to user input operate correctly.}}\\
 \multicolumn{2}{p{14cm}}{Lists operate correctly and ignore  selection events.}\\
 \hline
 \multicolumn{2}{p{14cm}}{\textbf{4. Verify that all components respond quickly to user input.}}\\
 \multicolumn{2}{p{14cm}}{n/a.}\\
 \hline
 \multicolumn{2}{p{14cm}}{\textbf{5. Verify that component actions and functionality do not degrade with extreme use (sustained rapid input, large numbers of input changes, etc).}}\\
 \multicolumn{2}{p{14cm}}{Large numbers of states and transition entries remain readable using scroll bars.}\\
 \hline
 \multicolumn{2}{p{14cm}}{\textbf{6. Verify that all data displayed within the element is visible, readable, correctly arranged and correctly labelled.}}\\
 \multicolumn{2}{p{14cm}}{Known states are readable and clear. Transitions are readable but no always clear, due to the layout of the time-stamp. The clarity could be improved with better formatting.}\\
 \hline
 \multicolumn{2}{p{14cm}}{\textbf{7. Verify that the element behaves sensibly when window resizing occurs, and that it remains usable and data remains visible whenever possible.}}\\
 \multicolumn{2}{p{14cm}}{resizing has no detrimental effect on the tab. Scroll bars are available when necessary.}\\
 \hline
 \multicolumn{2}{p{14cm}}{\textbf{8. Verify that the element updates promptly when responding to changes in data.}}\\
 \multicolumn{2}{p{14cm}}{New states and transitions are displayed immediately.}\\
 \hline
 \textbf{Fixes Required} & Re-format the state transition list entries to improve clarity.\\
 \bottomrule
\end{longtable}
\clearpage

\begin{longtable}{ l p{10cm} }
 \hline
 \multicolumn{2}{c}{\textbf{IR Data Tab}}\\
 \hline
 \textbf{Purpose} & Displays the IR sensor data for the selected robot.\\
 \textbf{Required Functionality} & Must provide a visual display of the selected robot's IR sensor values. Must provide a numerical display of the robot's IR sensor values. Must support both active and background values. Must update immediately when new values arrive.\\
 \hline
 \multicolumn{2}{p{14cm}}{\textbf{1. Examine UI element visually. Verify that it appears correct. Verify that it contains all elements necessary to satisfy its purpose.}}\\
 \multicolumn{2}{p{14cm}}{The element is lacking clear headings for some of the data points. The bar graph is visually correct. The numerical displays are correct but their current format takes up a lot of space.}\\
 \hline
 \multicolumn{2}{p{14cm}}{\textbf{2. Examine all text within the element. Check for errors in both meaning and spelling.}}\\
 \multicolumn{2}{p{14cm}}{All spelling correct. Some headings have unclear meanings.}\\
 \hline
 \multicolumn{2}{p{14cm}}{\textbf{3. Verify that all components within the element which perform actions in response to user input operate correctly.}}\\
 \multicolumn{2}{p{14cm}}{n/a.}\\
 \hline
 \multicolumn{2}{p{14cm}}{\textbf{4. Verify that all components respond quickly to user input.}}\\
 \multicolumn{2}{p{14cm}}{n/a.}\\
 \hline
 \multicolumn{2}{p{14cm}}{\textbf{5. Verify that component actions and functionality do not degrade with extreme use (sustained rapid input, large numbers of input changes, etc).}}\\
 \multicolumn{2}{p{14cm}}{Sensor values that are out of range are not displayed.}\\
 \hline
 \multicolumn{2}{p{14cm}}{\textbf{6. Verify that all data displayed within the element is visible, readable, correctly arranged and correctly labelled.}}\\
 \multicolumn{2}{p{14cm}}{All data is readable and correctly arranged. The IR data is not clearly labelled.}\\
 \hline
 \multicolumn{2}{p{14cm}}{\textbf{7. Verify that the element behaves sensibly when window resizing occurs, and that it remains usable and data remains visible whenever possible.}}\\
 \multicolumn{2}{p{14cm}}{Some of the data becomes hidden if the window width is reduced.}\\
 \hline
 \multicolumn{2}{p{14cm}}{\textbf{8. Verify that the element updates promptly when responding to changes in data.}}\\
 \multicolumn{2}{p{14cm}}{New data is reflected immediately.}\\
 \hline
 \textbf{Fixes Required} & Add clear headings and reformat the numerical data to improve clarity. Rearrange the layout so that the bar graph fits within the minimum window width.\\
 \bottomrule
\end{longtable}
\clearpage

\begin{longtable}{ l p{10cm} }
 \hline
 \multicolumn{2}{c}{\textbf{Custom Data Tab}}\\
 \hline
 \textbf{Purpose} & Displays custom data reported by the selected robot.\\
 \textbf{Required Functionality} & Must display each custom data point in, including both key and value strings. Must update immediately when new data is received.\\
 \hline
 \multicolumn{2}{p{14cm}}{\textbf{1. Examine UI element visually. Verify that it appears correct. Verify that it contains all elements necessary to satisfy its purpose.}}\\
 \multicolumn{2}{p{14cm}}{The tab is visually correct, including a table to display the custom data key/value pairs.}\\
 \hline
 \multicolumn{2}{p{14cm}}{\textbf{2. Examine all text within the element. Check for errors in both meaning and spelling.}}\\
 \multicolumn{2}{p{14cm}}{All spelling correct and meanings clear.}\\
 \hline
 \multicolumn{2}{p{14cm}}{\textbf{3. Verify that all components within the element which perform actions in response to user input operate correctly.}}\\
 \multicolumn{2}{p{14cm}}{The table cannot be modified using the mouse or keyboard, as intended.}\\
 \hline
 \multicolumn{2}{p{14cm}}{\textbf{4. Verify that all components respond quickly to user input.}}\\
 \multicolumn{2}{p{14cm}}{n/a.}\\
 \hline
 \multicolumn{2}{p{14cm}}{\textbf{5. Verify that component actions and functionality do not degrade with extreme use (sustained rapid input, large numbers of input changes, etc).}}\\
 \multicolumn{2}{p{14cm}}{Long keys and values and large numbers of key/value pairs are handled gracefully with scroll bars.}\\
 \hline
 \multicolumn{2}{p{14cm}}{\textbf{6. Verify that all data displayed within the element is visible, readable, correctly arranged and correctly labelled.}}\\
 \multicolumn{2}{p{14cm}}{All data is displayed clearly. The columns of the table are clearly labelled.}\\
 \hline
 \multicolumn{2}{p{14cm}}{\textbf{7. Verify that the element behaves sensibly when window resizing occurs, and that it remains usable and data remains visible whenever possible.}}\\
 \multicolumn{2}{p{14cm}}{The table scales to fit the available space, providing scroll bars whenever necessary.}\\
 \hline
 \multicolumn{2}{p{14cm}}{\textbf{8. Verify that the element updates promptly when responding to changes in data.}}\\
 \multicolumn{2}{p{14cm}}{New and updated custom data is displayed immediately.}\\
 \hline
 \textbf{Fixes Required} & None.\\
 \bottomrule
\end{longtable}
\clearpage

\begin{longtable}{ l p{10cm} }
 \hline
 \multicolumn{2}{c}{\textbf{Individual Visualisation Settings Dialogs}}\\
 \hline
 \textbf{Purpose} & Allow the user to change settings for the data visualisations.\\
 \textbf{Required Functionality} & Must display the settings for the selected visualisation type in a pop-up, modal window. Must allow the user to change these settings. Must provide the user with options for applying or cancelling the changes.\\
 \hline
 \multicolumn{2}{p{14cm}}{\textbf{1. Examine UI element visually. Verify that it appears correct. Verify that it contains all elements necessary to satisfy its purpose.}}\\
 \multicolumn{2}{p{14cm}}{All settings dialogs appear correctly, containing all the necessary controls to adjust the available settings.}\\
 \hline
 \multicolumn{2}{p{14cm}}{\textbf{2. Examine all text within the element. Check for errors in both meaning and spelling.}}\\
 \multicolumn{2}{p{14cm}}{All spelling correct and all meanings clear.}\\
 \hline
 \multicolumn{2}{p{14cm}}{\textbf{3. Verify that all components within the element which perform actions in response to user input operate correctly.}}\\
 \multicolumn{2}{p{14cm}}{All settings controls working correctly.}\\
 \hline
 \multicolumn{2}{p{14cm}}{\textbf{4. Verify that all components respond quickly to user input.}}\\
 \multicolumn{2}{p{14cm}}{Settings changes take effect immediately after pressing the apply button.}\\
 \hline
 \multicolumn{2}{p{14cm}}{\textbf{5. Verify that component actions and functionality do not degrade with extreme use (sustained rapid input, large numbers of input changes, etc).}}\\
 \multicolumn{2}{p{14cm}}{All components appear robust to repeated, rapid use. Input fields are validated to reject out of range values.}\\
 \hline
 \multicolumn{2}{p{14cm}}{\textbf{6. Verify that all data displayed within the element is visible, readable, correctly arranged and correctly labelled.}}\\
 \multicolumn{2}{p{14cm}}{All settings are clearly displayed, labelled and arranged.}\\
 \hline
 \multicolumn{2}{p{14cm}}{\textbf{7. Verify that the element behaves sensibly when window resizing occurs, and that it remains usable and data remains visible whenever possible.}}\\
 \multicolumn{2}{p{14cm}}{Dialog windows cannot be resized.}\\
 \hline
 \multicolumn{2}{p{14cm}}{\textbf{8. Verify that the element updates promptly when responding to changes in data.}}\\
 \multicolumn{2}{p{14cm}}{n/a.}\\
 \hline
 \textbf{Fixes Required} & None.\\
 \bottomrule
\end{longtable}
\clearpage

\begin{longtable}{ l p{10cm} }
 \hline
 \multicolumn{2}{c}{\textbf{General Visualiser Functionality}}\\
 \hline
 \textbf{Purpose} & Displays the live video feed and overlays the the data visualisations.\\
 \textbf{Required Functionality} & Must display the video feed. Must render the data visualisations based on the current data. Must allow the user to select a robot by clicking on it within the image.\\
 \hline
 \multicolumn{2}{p{14cm}}{\textbf{1. Verify that the video image is displayed correctly.}}\\
 \multicolumn{2}{p{14cm}}{The video feed image is displayed correctly, and updates at a decent rate. Due to the mounting orientation of the camera the image appears reversed when compared to the real world view. Settings could be added to allow the user to flip the image. }\\
 \hline
 \multicolumn{2}{p{14cm}}{\textbf{2. Verify that user clicks within the visualiser space are located correctly, at a number of different window sizes.}}\\
 \multicolumn{2}{p{14cm}}{User clicks are correctly located as shown by the cross-hairs showing the latest click location. Tested for a number of different sizes and image dimensions / aspect ratios, and worked correctly each time.}\\
 \hline
 \multicolumn{2}{p{14cm}}{\textbf{3. Verify that robots can be selected by clicking on their location in the visualiser image.}}\\
 \multicolumn{2}{p{14cm}}{Robots can be selected by clicking, with a reasonable tolerance. Robots positioned directly adjacent to one another can still be selected correctly. Clicking a robot with ID 10 caused a robot with ID 1 to be selected erroneously.}\\
 \hline
 \textbf{Fixes Required} & Add setting to allow the user to flip the video image. Robot selection works incorrectly for robots with IDs beginning with the same digit; determine the cause and fix.\\
 \bottomrule
\end{longtable}

As per test strategy B, each data visualisation type was then tested individually, obtaining the following results.
\clearpage

\begin{longtable}{ l p{10cm} }
 \hline
 \multicolumn{2}{c}{\textbf{Robot ID Visualisation}}\\
 \hline
 \textbf{Purpose} & Overlay the numerical ID of the robot on the video feed.\\
 \textbf{Required Functionality} & Display the numerical ID as a text string adjacent to the related robot.\\
 \textbf{Settings} & On / off (Toggle). Display for selected robot only or all robots (Toggle).\\
 \hline
 \multicolumn{2}{p{14cm}}{\textbf{1. Define input data and expected representation.}}\\
 \multicolumn{2}{p{14cm}}{Five active robots using ID's 0, 1, 5, 7 and 8. expect to see the ID numbers rendered to the left of each robot, slightly above their center.}\\
 \hline
 \multicolumn{2}{p{14cm}}{\textbf{3. Verify the representation is as expected.}}\\
 \multicolumn{2}{p{14cm}}{Numerical ID numbers appear next to each robot. The ID is positioned slightly too far to the left of each robot, sometimes overlapping with other visualisations in crowded areas. Settings apply correctly.}\\
 \hline
 \multicolumn{2}{p{14cm}}{\textbf{4. Check that the visualisation is clear and any text is legible.}}\\
 \multicolumn{2}{p{14cm}}{ID numbers are legible.}\\
 \hline
 \multicolumn{2}{p{14cm}}{\textbf{5. Repeat for multiple sets of input data.}}\\
 \multicolumn{2}{p{14cm}}{Tested with five robots.}\\
 \hline
 \multicolumn{2}{p{14cm}}{\textbf{6. Verify that integrity is maintained with extreme data, corner cases and zero data, wherever possible.}}\\
 \multicolumn{2}{p{14cm}}{Robot IDs higher than 999 result in long strings, which overlap with the robot itself and its position/direction visualisation. Robot swarm of this size are not anticipated however.}\\
 \hline
 \multicolumn{2}{p{14cm}}{\textbf{7. Verify that integrity is maintained at a range of window sizes, within reasonable limits.}}\\
 \multicolumn{2}{p{14cm}}{IDs display at the same size for all visualiser sizes, therefore remaining legible but occupying excessive space at small sizes. However the visualiser is not usable at such small sizes, so this is not anticipated to cause issues.}\\
 \hline
 \textbf{Fixes Required} & Reposition IDs slightly closer to the robot. Potentially allow the user to configure the positioning.\\
 \bottomrule
\end{longtable}
\clearpage

\begin{longtable}{ l p{10cm} }
 \hline
 \multicolumn{2}{c}{\textbf{Robot Name Visualisation}}\\
 \hline
 \textbf{Purpose} & Overlay the name of the robot on the video feed.\\
 \textbf{Required Functionality} & Display the name of the robot as a text string adjacent to the related robot.\\
 \textbf{Settings} & On / off (Toggle). Display for selected robot only or all robots (Toggle).\\
 \hline
 \multicolumn{2}{p{14cm}}{\textbf{1. Define input data and expected representation.}}\\
 \multicolumn{2}{p{14cm}}{Five active robots reporting the names Robot\_0, Robot\_1, Robot\_5, Robot\_7 and Robot\_8 in watchdog packets. Expect to see the names rendered to the right of each robot, slightly above their center.}\\
 \hline
 \multicolumn{2}{p{14cm}}{\textbf{3. Verify the representation is as expected.}}\\
 \multicolumn{2}{p{14cm}}{Names appear next to each robot. The text is positioned correctly to the right of each robot. Settings apply correctly.}\\
 \hline
 \multicolumn{2}{p{14cm}}{\textbf{4. Check that the visualisation is clear and any text is legible.}}\\
 \multicolumn{2}{p{14cm}}{Names are legible.}\\
 \hline
 \multicolumn{2}{p{14cm}}{\textbf{5. Repeat for multiple sets of input data.}}\\
 \multicolumn{2}{p{14cm}}{Tested with five robots.}\\
 \hline
 \multicolumn{2}{p{14cm}}{\textbf{6. Verify that integrity is maintained with extreme data, corner cases and zero data, wherever possible.}}\\
 \multicolumn{2}{p{14cm}}{Extremely long names are displayed up to the edge of the image.}\\
 \hline
 \multicolumn{2}{p{14cm}}{\textbf{7. Verify that integrity is maintained at a range of window sizes, within reasonable limits.}}\\
 \multicolumn{2}{p{14cm}}{Names remain legible for all usable sizes of the visualiser.}\\
 \hline
 \textbf{Fixes Required} & None.\\
 \bottomrule
\end{longtable}
\clearpage

\begin{longtable}{ l p{10cm} }
 \hline
 \multicolumn{2}{c}{\textbf{Robot State Visualisation}}\\
 \hline
 \textbf{Purpose} & Overlay the current state of the robot on the video feed.\\
 \textbf{Required Functionality} & Display the state of the robot as a text string adjacent to the related robot. Update this display whenever the state changes.\\
 \textbf{Settings} & On / off (Toggle). Display for selected robot only or all robots (Toggle).\\
 \hline
 \multicolumn{2}{p{14cm}}{\textbf{1. Define input data and expected representation.}}\\
 \multicolumn{2}{p{14cm}}{Five active robots oscillating between STATE1 and STATE2. Expect to see the states rendered to the right of each robot, below the name visualisation.}\\
 \hline
 \multicolumn{2}{p{14cm}}{\textbf{3. Verify the representation is as expected.}}\\
 \multicolumn{2}{p{14cm}}{States appear next to each robot. The text is positioned correctly to the right of each robot and below the name text. Settings apply correctly.}\\
 \hline
 \multicolumn{2}{p{14cm}}{\textbf{4. Check that the visualisation is clear and any text is legible.}}\\
 \multicolumn{2}{p{14cm}}{States are legible.}\\
 \hline
 \multicolumn{2}{p{14cm}}{\textbf{5. Repeat for multiple sets of input data.}}\\
 \multicolumn{2}{p{14cm}}{Tested with five robots, each changing state at different times.}\\
 \hline
 \multicolumn{2}{p{14cm}}{\textbf{6. Verify that integrity is maintained with extreme data, corner cases and zero data, wherever possible.}}\\
 \multicolumn{2}{p{14cm}}{Extremely long states are displayed up to the edge of the image.}\\
 \hline
 \multicolumn{2}{p{14cm}}{\textbf{7. Verify that integrity is maintained at a range of window sizes, within reasonable limits.}}\\
 \multicolumn{2}{p{14cm}}{States remain legible for all usable sizes of the visualiser.}\\
 \hline
 \textbf{Fixes Required} & None.\\
 \bottomrule
\end{longtable}
\clearpage

\begin{longtable}{ l p{10cm} }
 \hline
 \multicolumn{2}{c}{\textbf{Robot Position Visualisation}}\\
 \hline
 \textbf{Purpose} & Overlay a small circle around the robot's current position.\\
 \textbf{Required Functionality} & Render a circle outline around the robots current position. Use a thicker line to highlight the robot if it is currently selected.\\
 \textbf{Settings} & On / off (Toggle).\\
 \hline
 \multicolumn{2}{p{14cm}}{\textbf{1. Define input data and expected representation.}}\\
 \multicolumn{2}{p{14cm}}{Five active robots moving around the arena in different paths. Expect to see a circle rendered around the center of each robot, updating as their positions change.}\\
 \hline
 \multicolumn{2}{p{14cm}}{\textbf{3. Verify the representation is as expected.}}\\
 \multicolumn{2}{p{14cm}}{Circles are correctly rendered for all robots, and update correctly over time. Can be correctly toggled on and off.}\\
 \hline
 \multicolumn{2}{p{14cm}}{\textbf{4. Check that the visualisation is clear and any text is legible.}}\\
 \multicolumn{2}{p{14cm}}{Circles are clear.}\\
 \hline
 \multicolumn{2}{p{14cm}}{\textbf{5. Repeat for multiple sets of input data.}}\\
 \multicolumn{2}{p{14cm}}{Tested with five robots.}\\
 \hline
 \multicolumn{2}{p{14cm}}{\textbf{6. Verify that integrity is maintained with extreme data, corner cases and zero data, wherever possible.}}\\
 \multicolumn{2}{p{14cm}}{Circles display correctly for all positions within the image.}\\
 \hline
 \multicolumn{2}{p{14cm}}{\textbf{7. Verify that integrity is maintained at a range of window sizes, within reasonable limits.}}\\
 \multicolumn{2}{p{14cm}}{Circles render correctly at all sizes.}\\
 \hline
 \textbf{Fixes Required} & None.\\
 \bottomrule
\end{longtable}
\clearpage

\begin{longtable}{ l p{10cm} }
 \hline
 \multicolumn{2}{c}{\textbf{Robot Direction Visualisation}}\\
 \hline
 \textbf{Purpose} & Overlay a small line indicating the direction the robot is facing.\\
 \textbf{Required Functionality} & Render a line from the center of the robot outwards, in the direction it is facing. Use a thicker line if the robot is selected.\\
 \textbf{Settings} & On / off (Toggle).\\
 \hline
 \multicolumn{2}{p{14cm}}{\textbf{1. Define input data and expected representation.}}\\
 \multicolumn{2}{p{14cm}}{Five active robots moving around the arena in different paths. Expect to see a line rendered from the center of each robot outward in the direction it is facing, updating as its orientation changes.}\\
 \hline
 \multicolumn{2}{p{14cm}}{\textbf{3. Verify the representation is as expected.}}\\
 \multicolumn{2}{p{14cm}}{Lines are correctly rendered for all robots, and update correctly over time. Can be correctly toggled on and off.}\\
 \hline
 \multicolumn{2}{p{14cm}}{\textbf{4. Check that the visualisation is clear and any text is legible.}}\\
 \multicolumn{2}{p{14cm}}{Lines are clear.}\\
 \hline
 \multicolumn{2}{p{14cm}}{\textbf{5. Repeat for multiple sets of input data.}}\\
 \multicolumn{2}{p{14cm}}{Tested with five robots.}\\
 \hline
 \multicolumn{2}{p{14cm}}{\textbf{6. Verify that integrity is maintained with extreme data, corner cases and zero data, wherever possible.}}\\
 \multicolumn{2}{p{14cm}}{Lines display correctly for all orientations, at all positions within the image.}\\
 \hline
 \multicolumn{2}{p{14cm}}{\textbf{7. Verify that integrity is maintained at a range of window sizes, within reasonable limits.}}\\
 \multicolumn{2}{p{14cm}}{Lines render correctly at all sizes.}\\
 \hline
 \textbf{Fixes Required} & None.\\
 \bottomrule
\end{longtable}
\clearpage

\begin{longtable}{ l p{10cm} }
 \hline
 \multicolumn{2}{c}{\textbf{IR Data Visualisation}}\\
 \hline
 \textbf{Purpose} & Overlay a graphical representation of the robots infra-red sensor data.\\
 \textbf{Required Functionality} & IR sensor data represented in one of two modes. In proximity mode, display a line in the direction of each relevant sensor with a length inversely related to the sensors value, indicating an approximation of proximity. In `heat' mode, display a small box for each sensor adjacent to the robot and positioned to match the sensor layout, that changes colour as the sensor value changes. Uses the position and orientation of the robot to render with the correct position and rotation.\\
 \textbf{Settings} & On / off (Toggle). Display for selected robot only or all robots (Toggle). Proximity or heat mode (Toggle). Angle for each sensor in degrees (Numerical).\\
 \hline
 \multicolumn{2}{p{14cm}}{\textbf{1. Define input data and expected representation.}}\\
 \multicolumn{2}{p{14cm}}{Five active robots reporting their IR sensor data whilst an object is placed at varying distances from each of their sensors. Expect to see the graphical representations change to reflect the changing value.}\\
 \hline
 \multicolumn{2}{p{14cm}}{\textbf{3. Verify the representation is as expected.}}\\
 \multicolumn{2}{p{14cm}}{IR data is displayed in both modes. The display varies correctly as the sensor values change. Proximity lines change length correctly, but extend much further than necessary for low values. The boxes in heat mode change colour correctly, but are coloured close to black for low values, which does not display well on a dark background.}\\
 \hline
 \multicolumn{2}{p{14cm}}{\textbf{4. Check that the visualisation is clear and any text is legible.}}\\
 \multicolumn{2}{p{14cm}}{Both visualisations are clear.}\\
 \hline
 \multicolumn{2}{p{14cm}}{\textbf{5. Repeat for multiple sets of input data.}}\\
 \multicolumn{2}{p{14cm}}{Tested with 5 robots across a range of sensor values.}\\
 \hline
 \multicolumn{2}{p{14cm}}{\textbf{6. Verify that integrity is maintained with extreme data, corner cases and zero data, wherever possible.}}\\
 \multicolumn{2}{p{14cm}}{Out of range sensor data is not displayed.}\\
 \hline
 \multicolumn{2}{p{14cm}}{\textbf{7. Verify that integrity is maintained at a range of window sizes, within reasonable limits.}}\\
 \multicolumn{2}{p{14cm}}{The size of the visualisation does not change with the size of the image. However the proximity lines can only be representative, so their actual size is not important, only the relative variation in that size.}\\
 \hline
 \textbf{Fixes Required} & Set a more sensible maximum length for the proximity lines. Improve mapping between sensor values and line length (non-linear). Adjust heat mode colour scheme for clarity.\\
 \bottomrule
\end{longtable}
\clearpage

\begin{longtable}{ l p{10cm} }
 \hline
 \multicolumn{2}{c}{\textbf{Robot Path Visualisation}}\\
 \hline
 \textbf{Purpose} & Display the robots recent movement in the form of a trail.\\
 \textbf{Required Functionality} & Render the robot's position history as a sequence of line segments. support an adjustable sampling interval.\\
 \textbf{Settings} & On / off (Toggle). Display for selected robot only or all robots (Toggle). Sampling interval (Numerical).\\
 \hline
 \multicolumn{2}{p{14cm}}{\textbf{1. Define input data and expected representation.}}\\
 \multicolumn{2}{p{14cm}}{Five active robots moving around the arena along different paths. Expect to see a trail line behind each robot showing the path it has taken.}\\
 \hline
 \multicolumn{2}{p{14cm}}{\textbf{3. Verify the representation is as expected.}}\\
 \multicolumn{2}{p{14cm}}{Trails are correctly drawn for all robots, accurate to their approximate path. Varying the sample interval allows for longer, lower resolution and shorter, higher resolution paths. Can be correctly toggled on and off, and set to only display for the selected robot.}\\
 \hline
 \multicolumn{2}{p{14cm}}{\textbf{4. Check that the visualisation is clear and any text is legible.}}\\
 \multicolumn{2}{p{14cm}}{Trails are clearly drawn.}\\
 \hline
 \multicolumn{2}{p{14cm}}{\textbf{5. Repeat for multiple sets of input data.}}\\
 \multicolumn{2}{p{14cm}}{Tested with 5 robots moving along a number of different paths.}\\
 \hline
 \multicolumn{2}{p{14cm}}{\textbf{6. Verify that integrity is maintained with extreme data, corner cases and zero data, wherever possible.}}\\
 \multicolumn{2}{p{14cm}}{Setting an excessively large sampling interval leads to a very long, jagged path, as expected.}\\
 \hline
 \multicolumn{2}{p{14cm}}{\textbf{7. Verify that integrity is maintained at a range of window sizes, within reasonable limits.}}\\
 \multicolumn{2}{p{14cm}}{Paths remain accurate for all visualiser sizes, as path point positions are stored as proportional coordinates.}\\
 \hline
 \textbf{Fixes Required} & Add an upper limit to the sampling interval setting.\\
 \bottomrule
\end{longtable}
\clearpage



\begin{longtable}{ l p{10cm} }
 \hline
 \multicolumn{2}{c}{\textbf{Custom Data Visualisation}}\\
 \hline
 \textbf{Purpose} & Display a specific element of the robot's custom data as text.\\
 \textbf{Required Functionality} & Must display the key and the current value for the target data point as text adjacent to the robot, below the state text. Must update whenever the value for that key changes. The user must be able to set the target key.\\
 \textbf{Settings} & On / off (Toggle). Display for selected robot only or all robots (Toggle). Set the target data point (Text input).\\
 \hline
 \multicolumn{2}{p{14cm}}{\textbf{1. Define input data and expected representation.}}\\
 \multicolumn{2}{p{14cm}}{Five robots, each reporting custom data for two keys, with the values varying over time. Expect to see the target data key and value displayed as text below the state text.}\\
 \hline
 \multicolumn{2}{p{14cm}}{\textbf{3. Verify the representation is as expected.}}\\
 \multicolumn{2}{p{14cm}}{Custom data for the target key is correctly displayed, and updates immediately when new data arrives. The target key can be changed and the visualisation updates to reflect this. Can be toggled on and off correctly.}\\
 \hline
 \multicolumn{2}{p{14cm}}{\textbf{4. Check that the visualisation is clear and any text is legible.}}\\
 \multicolumn{2}{p{14cm}}{Text is clear and legible.}\\
 \hline
 \multicolumn{2}{p{14cm}}{\textbf{5. Repeat for multiple sets of input data.}}\\
 \multicolumn{2}{p{14cm}}{Tested with 5 robots reporting data for two different keys.}\\
 \hline
 \multicolumn{2}{p{14cm}}{\textbf{6. Verify that integrity is maintained with extreme data, corner cases and zero data, wherever possible.}}\\
 \multicolumn{2}{p{14cm}}{If no key is set, no data is displayed. Very long data values are displayed up to the edge of the image. If no data exists for the target key and the selected robot no data is displayed.}\\
 \hline
 \multicolumn{2}{p{14cm}}{\textbf{7. Verify that integrity is maintained at a range of window sizes, within reasonable limits.}}\\
 \multicolumn{2}{p{14cm}}{Remains visible at all usable visualiser sizes.}\\
 \hline
 \textbf{Fixes Required} & None.\\
 \bottomrule
\end{longtable}
\clearpage

\subsection{Fixes Implemented}
The results of the user interface testing highlighted no major problems, but did identify a number of smaller bugs and aesthetic issues. The following fixes were implemented:

\begin{itemize}
 \item The settings tabs in the visualiser panel were renamed to to better reflect their purpose.
 \item Long directory paths in the logging tab were truncated to only display the final 25 characters.
 \item The ordering of messages in the console were reversed, and now read from top to bottom in order.
 \item The redundant heading was removed from the overview tab.
 \item Entries in the state transition list were reformatted to clearly separate the time-stamp from the states.
 \item The IR data tab was reorganised to be more space efficient, and display data more clearly. Headings and labelled were added to further improve clarity.
 \item Code was added to flip the video image, a setting was added to the camera settings tab to enable and disable this feature.
 \item Robot IDs rendered in the visualiser were positioned slightly closer to the robots postiion.
 \item The IR data visualisation in proximity mode was adjusted to have a shorter maximum line length, and to use a non-linear mapping to improve the proximity approximation.
 \item The IR data visualisation in heat mode was adjusted to use white as the base colour for clarity, changing to red and increasing in size with increasing sensor value.
 \item An upper limit was added to the robot path sampling interval.
\end{itemize}

%----------------------------------------------------------------------------------------

\section{Data Model and Back End Unit Testing} \label{BackEndUnitTesting}
The next large code component requiring testing was the data model, which formed the majority of the application back-end code. Testing the data model manually, by inputting data packets and verifying the correct insertion of data into the model, was deemed to be too time consuming. This manual approach also posed another problem, in that the contents of the data model could only be examined through the user interface, which meant that the data model testing would be inherently coupled to the UI. This would make it harder to determine the source of any bugs found, as they might be related to the data model or the UI. Avoiding this kind of interdependency is part of the reason for adopting an object oriented approach to the software design and implementation. In order to avoid this issue, the back-end was tested using an automated, 'unit-testing' based approach.

Unit testing is a commonly used technique in professional software testing, and involves writing specific pieces of code which test individual 'units' of code. These test cases will manipulate the unit in some way, using the data and functions it exposes. Then the test case will assert a fact that should be true after the manipulation, such as a comparison of a data point within the unit and the value it should have following the given operations. Each test might include many of these assertion statements, and the test only passes provided that all assertions are determined to be true. In order to apply this technique to this system an extra class - `\textit{TestingWindow, testingwindow.cpp / .h}' - was added to the application. Alongside this class a number of individual test case functions were added. Each of these functions was written to test the data model in a specific way. Table \ref{tab:DataModelTestCases} lists the test case functions and their purposes.

\begin{longtable}{ l p{10cm} }
\caption[Data Model Test Cases]{Test cases used to unit-test the data model.}\\
 Test Case & Purpose and Method\\
 \hline
 Robot Insertion Test & Tests whether data objects for new robots are inserted into the model correctly. Supplies a packet of each possible type, using a new robot ID each time. After each packet asserts that the number of robots stored in the model has increased by one. Supplies another set of packets, this time reusing the existing robot IDs. Asserts after each packet that the number of robots stored in the model has not increased.\\
 Name Data Test & Tests whether data describing the name of a robot, received in watchdog packets, is inserted into the data model correctly. Supplies three watchdog packets, each with a different robot ID and robot name. Asserts that the data model now contains three robots, and that their name data matches the names entered in the packets. Supplies three new watchdog packets for the same set of robot IDs, with different names. Asserts that the data for each of the robots has been updated to include the new name data.\\
 State Data Test & Tests whether data describing the current state of a robot, received through state packets, is inserted into the data model correctly. Supplies three state packets, each with a different robot ID and state. Asserts that the data model now contains three robots, and that their state data matches the states entered via the packets. Supplies three new state packets for the same set of robot IDs, with different states. Asserts that the state data for each of the robots in the model now matches the new states.\\
 Position Data Test & Tests whether packets describing the current position and orientation of a robot are correctly parsed and the data stored correctly in the data model. Supplies three position packets, each with a different robot ID and different position values. Asserts that the data model now contains data for three robots. Asserts that this data matches the values in the packets for x-position, y-position and angle individually. Supplies three more position packets for the same set of robot IDs with new position and angle values. Asserts that the data in the model for each robot has been updated to reflect the new values. Floating point number assertions are done using a tolerance comparison, with a tolerance of $ 1 \times 10^{-7} $.\\
 IR Data Test & Tests whether packets describing a robot's infra red sensor values are correctly parsed and the data correctly stored in the data model. Supplies an IR data packet, with each sensor reading containing a different value. Asserts that the IR data in the model matches each of the values in turn. Supplies another IR data packet with new, unique values. Asserts that the IR data in the model now matches each of the new values. This process is repeated for packets of background IR data type.\\
 Custom Data Test & Tests whether packets describing custom data key value pairs are correctly parsed and the data correctly stored in the data mode. Supplies three custom data packets, each with a different robot ID and a different value, for a single key. Asserts that the data model now contains data for three robots, and that each robot has a custom data entry for the given key. Asserts that the value for each of these entries matches the value supplied in the relevant packet. Supplies three new packets for the same set of robots with a new key and a new value. Asserts that each robot now contains custom data for the second key, and that the values match those supplied in the packets. Supplies three more packets using the original key, with new values. Asserts that the values for the original key for each robot have been updated to match the values in the latest packet set.\\
 Position History Test & Tests whether position data supplied for a robot is correctly sampled and stored in the position history portion of the data model. Sets the position history sampling rate to 2. Supplies twenty position packets, all attributed to the same robot ID. Asserts that the position history now contains ten entries. Asserts that the x and y-positions values for each of these entries match the values in every second packet, in reverse order. Floating point number assertions are done using a tolerance comparison, with a tolerance of $ 1 \times 10^{-7} $.\\
 State History Test & Tests whether state the state transition history data is correctly formed from a sequence of state packets. Supplies five state packets, each attributed to the same robot ID and containing different states. Asserts that the state transition history now contains five entries. Asserts that the entries describe the correct state transitions, as described by the packets, in reverse order.\\
 Bad Data Test & Tests whether badly formed data packets are correctly rejected by the data model. Supplies a number of correctly formed data packets, each with a different robot ID, and asserts that the data model now contains the correct number of robot entries. Supplies a number of invalid and malformed data packets, distributed between the already used robot IDs and several new IDs. Asserts that the number of robots in the model has not changed, and that the data in each of the existing robots has not changed.\\
 \bottomrule\\
	
 \label{tab:DataModelTestCases}
\end{longtable}

The TestingWindow class provides an additional user interface window, which can be opened by selecting the `\textit{Testing Window}' option from the developer menu on the main tool-bar. This window displays a list of the available tests, a text area for displaying test results, and controls for running either a single test or the full set. Each time a test is run the class instantiates a new data model object, performs the operations for the test in question, displaying the steps involved as text in the results window. For each assertion the text describes what is being asserted and states whether the result was true or false. The overall result of each test is then stated at the end, and the data model object is destroyed. This therefore allows any developer working on the system to open this window at any time whilst the application is running and verify that the full set of tests still passes. This can be done whenever changes are made to the back end code, and gives the developer a degree of certainty that the data model is still functioning correctly. New tests can be easily added by implementing a new test function and adding it as a test case. The interface for this test window is shown in figure .

[TEST WINDOW UI SCREENSHOT]

\subsection{Results}
In its final state following this project the data model was implemented such that all of the stated test cases passed successfully when run. This indicated that the data model was implemented to a satisfactory standard, and that if data was supplied in correctly formatted packets, it would be correctly stored in the model. Other application functionality that relied on the data model could therefore be used and tested with the assumption that the model was operating correctly.

\subsection{Issues with this Approach}
The unit testing approach suffers from a number of issues. First and foremost the results of the tests are only as good as the tests themselves - meaning that any bugs in the code of each test could be misinterpreted as bugs in the software itself. To mitigate this the tests are designed to be as logically simple as possible, and perform the smallest number of operations necessary to achieve the behaviour under test. This minimises the chances of a mistake in the implementation of the test code. 

This approach also relies on the developer writing the tests correctly determining and entering the required result for each assertion statement. One example of this issue is in floating point comparison. The accuracy with which a floating point variable can describe a number is inherently limited due to the way a floating point variable is constructed. In almost all cases a floating point number will differ from the exact value it is attempting to represent by some small amount. This can present an issue when performing comparisons between the contents of a floating point variable and an exact value, leading to false negatives. In order to avoid this all floating point comparisons have been done using a threshold, rather than a direct comparison. This technique asserts that when the expected value is subtracted from the variable being tested, the modulus of the result is less than some very small tolerance value, indicating that the value in the variable is within an acceptable range of the expected value. A tolerance value of $ 1 \times 10^{-7} $ was used in all test cases, as this was deemed to indicate sufficient accuracy for the data in this system. To give some perspective to this number note that the x-position value of a robot is stored as a floating point value between 0 and 1, representing a portion of a physical distance of approximately two and a half meters. A discrepancy of +/- $  1 \times 10^{-7} $ therefore equates to an error of $ 2.5 \mu m $. Hence this tolerance value indicates more than adequate accuracy.

%----------------------------------------------------------------------------------------

\section{Validation Testing} \label{VerificationTesting}


%----------------------------------------------------------------------------------------

\section{Evaluation} \label{Evaluation}

\subsection{Method}
\subsection{Results}
\subsection{Analysis}

%----------------------------------------------------------------------------------------